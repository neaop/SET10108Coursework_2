\documentclass[journal,transmag]{IEEEtran}

\usepackage{cite}
\usepackage[pdftex]{graphicx}
% declare the path(s) where your graphic files are
\graphicspath{Figures/}
\DeclareGraphicsExtensions{.pdf,.jpeg,.png,.jpg}
\usepackage{amsmath}
\interdisplaylinepenalty=2500
\usepackage{algorithmic}
\usepackage{array}
\usepackage[caption=false,font=normalsize,labelfont=sf,textfon =sf]{subfig}
\usepackage{dblfloatfix}
\usepackage{url}
\usepackage{lipsum}
\usepackage{xcolor}
\usepackage{listings}

\lstset{
	escapeinside={/*@}{@*/},
	language=Java,	
	basicstyle=\fontsize{8.5}{12}\selectfont,
	numbers=left,
	numbersep=2pt,    
	xleftmargin=2pt,
	frame=tb,
	columns=fullflexible,
	showstringspaces=false,
	tabsize=4,
	keepspaces=true,
	showtabs=false,
	showspaces=false,
	morekeywords={inline,public,class,private,protected,struct},
	captionpos=b,
	lineskip=-0.4em,
	aboveskip=10pt,
	extendedchars=true,
	breaklines=true,
	prebreak = \raisebox{0ex}[0ex][0ex]{\ensuremath{\hookleftarrow}},
	keywordstyle=\color[rgb]{0,0,1},
	commentstyle=\color[rgb]{0.133,0.545,0.133},
	stringstyle=\color[rgb]{0.627,0.126,0.941},
}

% correct bad hyphenation here


\begin{document}

\title{An Investigation into the Speed-up of a Ray Tracer Application via the use of OpenMP and MPI}

\author{\IEEEauthorblockN{Sam Dixon \\ 40056761@live.napier.ac.uk}
\IEEEauthorblockA{SET10108 - Concurrent and Parallel Systems \\ School of Computing,
Edinburgh Napier University, Edinburgh}% <-this % stops an unwanted space

\thanks{Project available at: github.com/neaop/SET10108Coursework\_2}}


\markboth{SET10108 - Coursework 2}{}
% The only time the second header will appear is for the odd numbered pages after the title page when using the twoside option.

\IEEEtitleabstractindextext{
\begin{abstract}
	The abstract goes here.
\end{abstract}

\begin{IEEEkeywords}
	C++11, Ray-Tracer, Parallel, OpenMP, MPI, Thread, Distribution, Speed Up.
\end{IEEEkeywords}}

\maketitle

\IEEEdisplaynontitleabstractindextext

\IEEEpeerreviewmaketitle

\section{Introduction}
 
	\IEEEPARstart{T}{he} aim of this report is to document and analyse the results of an attempt to speed-up a C++11 Ray tracer application - via the use of concurrency and parallelisation techniques. The methods being tested in this project are the \texttt{OpenMP} threading library and the \texttt{MPI} distribution library.
	
	\subsection{Ray Tracing}
		Ray Tracing a rendering technique that allows an image to be generated by tracing the path of a ray of light and simulating the effects that a virtual environment have upon it.
	
	\subsection{\texttt{OpenMP}}
		\texttt{OpenMP} is an open library that allows developers to parallelise sequential programs with minimal effort. 
	
	\subsection{\texttt{MPI}}
		\texttt{MPI} or Message Passing Interface is a method of distributed parallelism that operates by having multiple processors communicate by sending and receiving signals from one another via channels. 
		
\section{Methodology}
	
	
	\subsection{Profiling}
		Prior to implementing any methods that will save time, the sequential code must first be analysed. By using the Visual Studio Performance Profiler, it is possible to evaluate the sequential code and locate the functions or methods that use up the most CPU time. Once the potentially problematic areas have been identified, a suitable parallelisation method can be implemented to reduce the impact of those areas on the execution time. It should be noted that all code presented in the report was run with all compiler optimisation turned off. 	
		
	\subsection{Data Collection}
		To ensure fair comparison and accurate results, each implementation was tested using the same parameters. Each solution was run for two-thousand and forty generations and the execution time was recorded. This was then repeated one-hundred times for each application and the results were then averaged. The various techniques were all tested using four threads, then again using eight threads to investigate if the use of virtual cores would have an impact upon the speed of execution. All code was benchmarked on the same PC, the specifications of which can be seen in Table \ref{pcSpecsTabel}, Page \pageref{pcSpecsTabel}. As well as the average execution time, speed-up and efficiency are calculated for each technique. Speed-up is defined as: 
		\[S=\frac{s_{t}}{p_{t}}\]
		With \(s_{t}\) being sequential time and \(p_{t}\) being parallel time.
		Efficiency is calculated as 
		\[E = \frac{S}{P}\]
		\(S\) being speed-up from the previous formula and \(P\) is the number of physical cores on the CPU.
		\begin{table}[]
			\centering
			\caption{PC Specifications}
			\label{pcSpecsTabel}
			\begin{tabular}{|l|l|}
				\hline
				CPU & i7-4790k 4 Core HT @ 4.00 ghz \\ \hline
				RAM & 16gb Dual Channel DDR3        \\ \hline
				GPU & Nvidia GeForce GTX 980        \\ \hline
				OS  & Windows 7 64 Bit              \\ \hline
			\end{tabular}
		\end{table}
		
	
	
\section{Results}
	
\section{Conclusion}
	
	\subsubsection{Future Work}
	
\appendices
\renewcommand\refname{Bibliography}
\bibliographystyle{IEEEtran}
\bibliography{Bibliography}
\nocite{Williams:1483005}
\end{document}